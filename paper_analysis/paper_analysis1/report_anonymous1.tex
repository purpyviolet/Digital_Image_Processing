\documentclass[a4paper]{article}

\input{style/ch_xelatex.tex}
\input{style/scala.tex}
\usepackage{caption} % 允许在非浮动环境中使用标题
\lstset{frame=, basicstyle={\footnotesize\ttfamily}}

\usepackage{graphicx} % 引入图片
\usepackage{listings} % 展示代码
\usepackage{xcolor} % 代码高亮
\usepackage{ragged2e}
% 配置代码展示的样式
\lstset{
  basicstyle=\ttfamily,
  keywordstyle=\color{blue},
  commentstyle=\color{green},
  stringstyle=\color{red},
  frame=single,
  numbers=left,
  numberstyle=\small,
  stepnumber=1,
  numbersep=5pt,
  tabsize=4,
  showspaces=false,
  showstringspaces=false
}

\graphicspath{ {images/} }
\usepackage{ctex}
%-----------------------------------------BEGIN DOC----------------------------------------

\title{“基于巴特沃斯低通滤波的直方图均衡化去雾算法”论文研究报告}
\author{ZYH,YQF}
\date{\today}

\begin{document}

\maketitle

\section{论文目的}
这篇论文的主要目的是设计和验证一种新的图像去雾算法,以应对雾霾天气导致的室外监控图像质量下降问题。雾和霾天气中的微小水滴或颗粒物散射和吸收光线,导致获取的图像模糊不清,细节丢失,影响图像的识别和分析。为了解决这一问题,提高图像在雾霾天气下的清晰度和可见度,该研究提出了一种基于巴特沃斯低通滤波的直方图均衡化去雾算法。

该研究的核心动机在于,常见的图像增强方法(如空域法和频域法)在处理雾天图像时,虽能提高图像对比度和动态范围,但常常伴随着图像细节的丢失。直方图均衡化是一种经典的图像增强方法,能有效提升图像的对比度,但也存在图像细节容易丢失的问题。因此,本文通过对直方图均衡化和巴特沃斯低通滤波理论的分析与结合,设计了一种新的去雾算法。

该算法的目的是通过改进图像增强技术,使室外监控系统在雾霾天气下获取的图像更加清晰,提高图像的可用性和识别率。通过将直方图均衡化方法与巴特沃斯低通滤波理论相结合,该算法旨在同时增强图像对比度和保留图像细节,克服了传统直方图均衡化方法在提升图像对比度时可能导致的细节丢失问题。

\section{方法}
论文采用了两种主要的方法组合来实现去雾算法,即直方图均衡化和巴特沃斯低通滤波。这两种方法的结合旨在提高雾天图像的清晰度和对比度,同时尽可能保留图像细节。以下是详细的方法介绍及其在去雾过程中的应用,直接引用原文内容进行说明。

1. 直方图均衡化算法:直方图均衡化算法的思想主要是利用图像灰度点的运算对图像的直方图进行变换来提高图像的对比度,扩展图像的动态范围。通过对直方图进行非线性拉伸,重新分配像素值,使得直方图均匀分布,从而实现图像对比度的增强。(摘自原文第1.1节)

2. 巴特沃斯低通滤波器:本文算法需要进行低通滤波,因此需选用滤波效果较为理想的滤波器。与理想低通滤波器相比,巴特沃斯低通滤波器保留成分和滤除成分之间的过渡平滑,因此不会产生如理想低通滤波器发生的振铃效应。巴特沃斯低通滤波是指通过算法滤除二维图像中的高频分量,从而得到低频分量。(摘自原文第1.3节)

结合这两种方法,该论文提出了一种新的去雾算法流程,其中包括对雾天图像先进行巴特沃斯低通滤波处理,分离图像的高频和低频分量,然后对低频分量执行直方图均衡化处理,最后将处理后的低频分量与高频分量重新组合,以获得去雾后的图像。这种方法的核心在于,通过巴特沃斯低通滤波器处理,减少图像的高频噪声,而直方图均衡化则用于增强图像的对比度和明亮度,两者的结合旨在在去雾的同时保留尽可能多的图像细节。

% 去掉figure环境,直接插入图像
\centering % 如果需要居中对齐图像
\includegraphics[width=0.8\linewidth]{images/flowchart.png}
\captionof{figure}{流程图} % 使用\captionof命令添加标题


\justifying
流程起始于图像的输入,首先判断图像是否为彩色。如果是彩色图像,它将经过RGB分解;如果不是,图像将直接进入灰度处理环节。对于RGB分解的结果,每个通道都将独立进行灰度处理。

接下来,处理后的灰度图像将进入滤波分频阶段,这一步骤将图像分为低频和高频信息。低频信息包含了图像的主要形状和轮廓,而高频信息则包含了边缘和细节。低频部分将接受直方图均衡化处理,目的是提高图像的对比度并增强可见性。

与此同时,处理高频信息的路径旨在保留图像细节。完成这两个并行处理后,增强的低频信息和保留的高频信息将重新结合,形成一幅经过均衡化处理的图像。如果原始输入是彩色图像,这时处理后的各个通道将重新组合成一幅RGB图像。

最终,这个综合处理后的图像被输出,标志着图像处理流程的结束。

\section{各步骤所得结果}
论文中提出的基于巴特沃斯低通滤波的直方图均衡化去雾算法主要通过以下步骤进行图像处理,以改善雾天图像的清晰度:

1. 滤波处理:首先对雾天图像进行滤波分频,这一步骤通过设置截止频率并构造滤波器完成,得到图像的低频分量。这一过程旨在分离图像的高频信息(如细节和边缘)和低频信息(如背景和主要形状)。

2. 分离高频和低频信息:接着,使用原图减去低频分量来得到高频分量,这样可以分别处理图像中的大范围信息和细节信息。通过这种分离,可以单独对低频分量进行处理,以增强图像的整体对比度,同时保留细节信息。

3. 直方图均衡化处理低频分量:对低频分量进行直方图均衡化处理,目的是通过扩展图像的动态范围来增强图像的对比度,使得图像在低频部分更加清晰。这一步骤改善了图像的整体可见性,尤其是在背景和主要形状方面。

4. 与高频分量相加:最后,将经过直方图均衡化处理的低频分量与高频分量进行线性相加,得到最终的去雾图像。这一步不仅增强了图像的对比度,也恢复了图像的细节信息,从而在提高图像清晰度的同时保留了更多的细节。

修改后的图片表现:

在滤波处理后,图像被有效地分离成低频和高频分量,为后续的处理打下了基础。
分离高频和低频信息后,使得可以针对性地处理图像的不同部分,保留细节的同时增强整体的可见度。
直方图均衡化处理低频分量后,图像的对比度得到了明显的提升,使得雾天图像的背景和主要形状变得更加清晰。
与高频分量相加后,最终的去雾图像在保留细节信息的同时,整体视觉效果明显改善,图像更加清晰,细节部分也得到了很好的恢复。
\section{图像质量评价方法}
该论文通过结合直方图均衡化和巴特沃斯低通滤波理论,提出了一种去雾算法,并采用了两种主要方法来评价图像质量:主观质量评价和客观质量评价。

主观质量评价分析
主观质量评价主要依赖于人的视觉感受来评价图像质量。在该论文中,作者通过视觉比较直方图均衡化算法和BHE(基于巴特沃斯低通滤波的直方图均衡化)算法处理后的雾天图像。评价的依据是图像的清晰度、细节保留情况以及整体视觉效果的提升。

客观质量评价分析
客观质量评价通过量化指标来评估图像质量,这些指标可以通过计算得出,不依赖于人的主观感受。该论文使用了两种客观质量评价指标:

1.图像信息熵:用于评价图像中包含的信息量。信息熵越高,图像含有的信息量越大,通常意味着图像质量越高。
  
2.峰值信噪比(PSNR):用于衡量图像处理前后的失真程度。PSNR值越高,表示图像失真越少,图像质量越好。

通过对比处理前后的图像信息熵和PSNR值,评价了直方图均衡化算法和BHE算法对雾天图像去雾效果的改善,以及图像质量的提升情况。

这种主客观结合的图像评价方法,既考虑了人的视觉感受,也依据了具体的量化指标进行评价,可以全面评估去雾算法对于提高雾天图像质量的效果

\section{论文算法复现}
在我们的研究中,为了验证基于巴特沃斯低通滤波的直方图均衡化去雾算法的有效性,我们采用了Python作为实验编程语言,利用了OpenCV和Numpy库来处理图像。实验的主要目标是对雾天图像进行清晰度和对比度的增强,以改善视觉效果和信息的可识别性。

具体实验步骤如下:

1. 图像读取与灰度转换:首先,我们使用OpenCV的`cv2.imread`函数读取雾天图像,然后将其转换为灰度图像,以便于进一步处理。

2. 巴特沃斯低通滤波:接下来,对灰度图像应用巴特沃斯低通滤波器,这一步骤旨在分离图像中的低频分量,即图像中的主要形状和轮廓信息。通过设置合适的截止频率和滤波器阶数,我们能够去除或减少高频噪声,同时保留图像的基本结构。

3. 直方图均衡化处理:将巴特沃斯低通滤波器处理后得到的低频分量进行直方图均衡化,该过程通过增强图像的局部对比度来提高图像的整体可见度和清晰度。

4. 图像重组:最后,将直方图均衡化后的低频分量与原始图像中的高频分量重新结合,从而获得最终的去雾图像。这一步骤确保了图像细节的恢复,同时增强了图像的对比度和清晰度。

通过以上步骤,我们成功实现了雾天图像的清晰度和对比度增强。下图展示了处理前后的图像对比:

% 去掉figure环境,直接插入图像
\centering % 如果需要居中对齐图像
\includegraphics[width=0.8\linewidth]{images/before_process.png}
\captionof{figure}{处理前的图像} % 使用\captionof命令添加标题

% 插入另一张图
\includegraphics[width=0.8\linewidth]{images/after_process.png}
\captionof{figure}{处理后的图像}

\justifying
实验结果表明,通过应用基于巴特沃斯低通滤波的直方图均衡化去雾算法,我们能够显著提高雾天图像的视觉质量,使原本模糊不清的图像变得更加清晰,细节信息得到了有效的恢复。这一结果验证了算法的有效性,为雾天图像处理提供了一种有效的技术途径。

\section{所得经验}
从论文中提出的基于巴特沃斯低通滤波的直方图均衡化去雾算法,我总结出以下经验:

1. 综合方法的重要性:单一的图像处理方法往往难以解决复杂的图像问题。本研究结合了直方图均衡化和巴特沃斯低通滤波两种技术,针对性地解决了雾天图像处理中的挑战。这说明在解决特定图像处理问题时,综合运用多种技术往往能达到更好的效果。

2. 图像分频处理的有效性:通过将图像分离为高频和低频分量,并分别处理,这种方法能够在增强图像对比度的同时保留重要的细节信息。这种分频处理思路对于其他图像增强任务也有借鉴意义。

3. 参数调整的重要性:论文中提到,截止频率的选择对滤波效果有显著影响。这表明在实际应用图像处理算法时,参数的调整对于算法性能有重要影响。适当的参数调整可以显著提高处理结果的质量。

4. 算法验证的必要性:论文通过与传统直方图均衡化算法的对比实验,验证了新算法的有效性。这说明在提出新的图像处理方法时,通过实验对比验证其相对于现有方法的改进是非常必要的。

5. 客观评价指标的作用:论文使用了图像信息熵和峰值信噪比等客观评价指标来评估去雾效果。这种客观评价方式为量化比较不同算法提供了依据,有助于准确理解算法的性能。

深入研究和实践这类综合图像处理方法,可以提升我们在图像处理领域的理论知识和实践技能,特别是在处理具有特定挑战(如雾天图像去雾)的应用场景时。此外,学习如何科学地选择和调整算法参数,以及如何通过实验设计和客观评价指标来验证算法效果,为我们以后从事图像处理研究和应用开发工作提供了宝贵经验。


\newpage
\section*{附件}

\subsection*{1.代码展示}
以下是本研究中使用的Python代码片段:

\begin{lstlisting}[language=Python]
import cv2
import numpy as np

def butterworth_lowpass_filter(d, d0, n):
    # d: distance matrix
    # d0: cutoff frequency
    # n: order of the filter
    return 1 / (1 + (d / d0) ** (2 * n))

def apply_histogram_equalization(image):
    # Apply histogram equalization to an image
    return cv2.equalizeHist(image)

# Load the image
image_path = 'foggy.png'
image = cv2.imread(image_path, cv2.IMREAD_GRAYSCALE) # Or cv2.IMREAD_COLOR for RGB

# Fourier transform
f = np.fft.fft2(image)
fshift = np.fft.fftshift(f)

# Create Butterworth low pass filter mask
rows, cols = image.shape
crow,ccol = rows//2 , cols//2
d = np.sqrt((np.arange(-crow, crow)**2).reshape(-1, 1) + (np.arange(-ccol, ccol)**2))
lowpass = butterworth_lowpass_filter(d, 500, 5) # d0=30, n=2 are chosen for demonstration

# Apply filter
fshift_filtered = fshift * lowpass
f_ishift = np.fft.ifftshift(fshift_filtered)
img_back = np.fft.ifft2(f_ishift)
img_back = np.abs(img_back)

# Histogram equalization on low-frequency component
low_freq_component = img_back.astype(np.uint8)
equalized_low_freq = apply_histogram_equalization(low_freq_component)

# High frequency component calculation and final image recombination might need
# additional steps, especially for handling the color images and combining the channels

# Display the original and the processed image
cv2.imshow('Original Image', image)
cv2.imshow('Processed Image', equalized_low_freq)
cv2.waitKey(0)
cv2.destroyAllWindows()

\end{lstlisting}

\end{document}
